
\section{How to use this?}

The source code of this project simultaneously defines
a report as a \LaTeX document;
as well as a Haskell library with an included test suite.
The latter can itself be used in two ways:
as a Haskell library purely for pattern manipulation,
or as a set of function definitions for the TidalCycles environment.

\subsection{Compiling the report}

To generate the PDF: clone the repository, open \texttt{report.tex} in your favorite \LaTeX editor, and compile.
You can also generate the report in a terminal by running
\texttt{pdflatex report; bibtex report; pdflatex report; pdflatex report}

If you don't have a \LaTeX environment installed locally, you can also upload the \emph{entire} repository to a service like Overleaf, set the main file to \texttt{report.tex}, and compile it there.

\subsection{Compiling and testing the library}

You should have stack installed (see \url{https://haskellstack.org/}) and
open a terminal in the same folder.

\begin{itemize}
  \item To compile everything: \verb|stack build|.
  \item To open ghci and play with the code: \verb|stack ghci|
  \item To run the tests: \verb|stack clean && stack test --coverage|
\end{itemize}

\subsection{Using the functions in TidalCycles compositions}

Getting the code to work in a live TidalCycles environment can be tricky.
Doing it properly involves following the following steps, which are quite general because the particulars will depend on which platform you are using:

\begin{itemize}
  \item Create a new stack project that pulls in both this library and TidalCycles as dependencies
  \item Install SuperCollider and SuperDirt locally
  \item Follow the rest of the installation instructions on the TidalCycles website, but use the local stack project instead of installing the package globally
  \item Create a \verb|BootTidal.hs| file somewhere which imports the \verb|morecycles| functions you want to use and binds them to the appropriate names
  \item Configure you reditor to read the \verb|BootTidal.hs| file
\end{itemize}

However, experience in our group has shown that this can be a time-consuming and error-prone process on some platforms (particularly: the commercial operating systems).
If the above steps are not working or take too long, we offer the following as a quick-and-dirty alternative:

\begin{itemize}
  \item Install TidalCycles, SuperCollider, and SuperDirt exactly as described on the TidalCycles website
  \item Select the Haskell source of the function(s) you want to use
  \item Remove all newlines and comments
  \item Paste these into the editor you configured for TidalCycles in the first step
  \item Use the keyboard shortcut you typically use for executing a line, to execute this code, adding the name binding to the current interpreter session
\end{itemize}
