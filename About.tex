\section{Introduction}

\subsection{About TidalCycles}

TidalCycles\footnote{\url{https://tidalcycles.org/}} is an environment for music livecoding:
a type of musical performance where an audience watches an artist write code which generates sound.
TidalCycles is based on Haskell and offers a ``mini-language'' for users to define cyclic patterns of ``events''.
Events, in turn, are time spans where a particular value is active.
In a musical context, these values can be the names of samples, notes, or chords;
but as there are no additional constraints put on event values,
they can also represent other parameters that vary over the course of the musical performance,
such as the cutoff frequency of a filter,
a boolean mask selecting certain notes and ignoring others,
etc.

In addition to the pattern mini-language,
TidalCycles also provides a large library of functions which can modify patterns or apply digital signal processing (DSP) effects to the generated audio.

Together, these provide a flexible creative platform for musicians to define an interesting piece of music declaratively, iterate on their ideas, and provide an entertaining experience for a live audience.

\subsection{About this project}

In this project,
each contributor was responsible for proposing, designing, implementing, and testing a pattern manipulation function:
one which takes (at least) one pattern as input and produces a pattern as output.
This structure allowed us to work more or less independently,
exploring each of our musical interests,
while still all learning how TidalCycles leverages the power of Haskell's inherant laziness
to work with infinite patterns and constrained randomness.
